\begin{abstract}
A package-management system consists of a client application working in conjunction with remote software repositories.
The community of SourceMod, a plugin loader for Source engine's game servers, lacks both of these imperative facets.
Having nearly all community-made plugins released in an arbitrary manner, a system must be put in place to abstract and standardize their distribution.
It needs to utilize recurring conventions in publishing to be compatible with a matured community with over a decade long development history.
The package manager on the client side can then use this system to provide its users the comfort of automatic installation, removal, dependency resolution and more.
\end{abstract}

{\small \textbf{Keywords}: SourceMod, Package manager, Server-client model, Automation}

\begin{sloppypar}

\renewcommand{\abstractname}{Streszczenie}
\begin{abstract}
System zarządzania pakietami składa się z aplikacji \mbox{klienckiej} współdziałającej ze zdalnymi repozytoriami pakietów.
W społeczności SourceMod, modyfikacji zarządzającej \mbox{rozszerzeniami} do serwerów gier w silniku Source, brakuje obu tych niezbędnych \mbox{aspektów}.
Posiadając prawie wszystkie wtyczki wypuszczone w arbitralny sposób, specjalny system musi być zbudowany w celu abstrakcji oraz standaryzacji ich dystrybucji.
Powinien on wykorzystywać powtarzające się konwencje w publikacji, aby być kompatybilnym z rozwiniętą społecznością z ponad dziesięcioletnią historią pracy.
Aplikacja kliencka jest wtedy w stanie użyć tego systemu, żeby zapewnić użytkownikom komfort automatycznej instalacji, usuwania, pobierania zależności, i nie tylko.
\end{abstract}

\end{sloppypar}

{\small \textbf{Słowa kluczowe}: SourceMod, System zarządzania pakietami, Klient-serwer, Automacja}
