\chapter{(TODO: name)}
\section{Source engine dedicated servers}

Valve's game engine, Source, is the engine behind many popular titles such as Counter-Strike: Global Offensive or Team Fortress 2.
It allows for running headless instances of any of its games, internally called \textit{mods}, through their Source Dedicated Server tool (srcds) on Linux, Windows and MacOS.
The server file structure is similar to one of client's:
\begin{itemize}
    \item \textbf{bin} -- TODO,
    \item \textbf{mod} -- abbreviated name of the mod, stores assets like maps or models,
    \item \textbf{mod/addon} -- directory for third-party extensions,
    \item \textbf{srcds} -- server instance binary.
\end{itemize}
Under \textit{addons}, each directory corresponds to a different extension, which are C++ libraries that modify the behavior of the server (TODO: cite valvesoftware wiki).
The Source engine exposes a set of interfaces they may interact with, for instance capture an event of a player connecting.
The caveats of this approach are twofold:
\begin{enumerate}
\item
The API that the Source engine provides is very bare bones and lacking features often required in servers of any kind, such as external database querying.
This leaves the developer to implement their own solutions, which may unnecessarily vary from one extension to another.
\item
The fact that extensions are natively built C++ libraries forces the developer to write and provide support for platform-dependent code.
\end{enumerate}
SourceMod seeks to alleviate these problems by wrapping the underlying C++ API with its own, exposed in a simple scripting language called SourcePawn.
Developing a SourceMod plugin allows programmers to extend the functionality of Source servers in OS-agnostic way.
On top of that, the new API provides them with more tools than just server behavior.
Developers are granted features such as a MySQL database driver or a parser for Valve-specific configuration files right from the base API.

\section{SourceMod as platform for extensibility}

Installing SourceMod is simply extracting its archive into \textbf{mod/addons/}, after which \textbf{mod/addons/sourcemod/} should look as follows:
\begin{itemize}
    \item \textbf{bin} -- SourceMod specific binaries,
    \item \textbf{cfg} -- configuration files for plugins or extensions,
    \item \textbf{plugins} -- plugin binaries,
    \item \textbf{extensions} -- extension binaries,
    \item \textbf{gamedata} -- text files storing game's variable memory offsets and function signatures to be shared across plugins for more complex extensibility capabilities,
    \item \textbf{scripting} -- plugin sources and the SourcePawn compiler,
    \item \textbf{translations} -- text files storing phrases for localization.
\end{itemize}

A plugin is a binary compiled by the SourcePawn compiler, it is independent of the operating system and handled exclusively by SourceMod.
An extension is required for low-level control which goes beyond the scope of plugins, which in itself is very rare.
Both plugins and extensions are loaded by SourceMod, but the latter are C++ libraries and thus system-dependent.
For example, a plugin is a menu pop-up in the game, while an extension is a database driver.

In relation to the aforementioned directory structure, SourceMod expects a sort of order, either by convention or API requirements.
A plugin loads localization only from \textbf{translations}, or function signatures from \textbf{gamedata}.
Ultimately, it is up to the plugin what to do and where to load its files from.
For instance, if a plugin comes bundled with assets, those usually are not placed in the SourceMod directory, but rather two directories up and across.
