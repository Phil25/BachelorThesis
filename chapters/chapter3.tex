\chapter{Solution}

\section{Current Publishing Conventions}

Although there is no set standard on how to publish plugins, conventions have emerged over time.
Taking advantage of them is the first step towards standardizing distribution.

There are three websites where either the important ones or the majority of plugins are found:
\begin{itemize}
    \item \textbf{AlliedModders forums} -- official forums of SourceMod, usually there exists a corresponding thread, regardless where the files are hosted,
    \item \textbf{GitHub} -- preferred by some developers, either to upload binaries directly to repositories or have them posted in the \textit{Releases} section,
    \item \textbf{LimeTech} -- third-party website hosting high-traffic SourceMod extensions, owned by one of the major contributors.
\end{itemize}

\subsection{AlliedModders forums}

The official forums are the main source of advertising for developers, thus a sure way to find plugins.
A thread dedicated to a release of one, either hosts the files as its attachments or instructs readers where to download them from.
All addons posted on the forums must adhere to the SourceMod license, GNU General Public License, version 3.
This license applies to derivative works as well, that is SourceMod plugins and extensions \cite{sourcemod-license}.
This means that source must be provided along with distribution.

Because of this the forums have a unique feature available to developers when posting their work.
If a file is the source of a plugin (\textit{.sp} extension), it will automatically be compiled by the online compiler and displayed next to the source attachment.
While convenient, it is quite limited.
Should a plugin make use of libraries outside of the standard, the compilation will fail.
In such cases, the developer needs to compile the plugin themselves and upload it as a separate attachment.
Each are given and identified by a unique ID, which is utilized for downloading.
Unfortunately, there is no API to parse them.

\subsection{GitHub}

GitHub is a preferable alternative to the forums as a file host by many developers.
For that, they mostly use the \textit{Releases} feature of the service but may store binary files along with everything else.
In case of the latter, there exists a URL which will always point to the latest version of any file in the repository.
A similar URL may exist pointing to a file in the \textit{Releases} section, but only for a specific release.
Therefor, if anyone wishes to programmatically fetch a file uploaded in the latest release, the GitHub API must be utilized.
It is then possible to iterate all the uploaded assets and read their names or URLs, among other properties.

\subsection{LimeTech}

LimeTech is not used by the public but rather a single developer, who also is a major contributor to both SourceMod and SourcePawn.
The website hosts files for many popular extensions, making it worthwhile to consider when finding the way to standardize distribution.

Like the forums and GitHub, it lacks a URL to download the latest publication of a project from.
And due to a lack of an API, one must resolve to scraping.
All the project releases hosted on the website are archives.
They are kept on pages of their own, in a three-column table of operating systems to specific versions.
A caveat here may be that a cell is empty, could be because a certain release for a specific system is not out yet, or is simply not supported.
Thus a traversal of the table is required to find the next, latest one available.

\subsection{Other websites}

\section{Plugin Metadata Database}
